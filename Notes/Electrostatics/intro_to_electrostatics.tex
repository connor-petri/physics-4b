\begin{center}
    \section{Introduction to Electrostatics}
    \hrulefill
    \subsection*{Charge}
\end{center}

\subsubsection*{Definition}

\hspace{.5cm}
Much like inertia, charge is a fundimental property of matter that describes how an object interacts with 
electric fields. It is measured in Coulombs $[C]$. Practically, charge indicates if the object has an excess 
or deficiency of electrons.

\subsubsection*{Quantization of Charge}
\hspace{.5cm}
Charge is quantized, meaning that it can only exist in discrete values. 
The smallest possible charge is the charge of an electron. Protons and electrons carry equal but opposite charges, 
represented by $e$.

\begin{center}
    $q_p = +e = 1.6 \times 10^{-19}C$\\
    $q_e = -e = -1.6 \times 10^{-19}C$
\end{center}

Much like energy and matter, charge is conserved in a closed system. This means that the total charge in a system
will remain constant. 

\begin{center}
    $\sum q_i = \sum q_f$
\end{center}

\hrulefill

\begin{center}
    \subsection*{Coloumb's Law}
\end{center}

\subsubsection*{Definition}
\hspace{.5cm}
Coulomb's law describes the fundimental force between 2 charged objects. It is given by the equation:

\begin{center}
    $F_e = k_e \frac{q_1q_2}{r^2} \Longrightarrow \vec{F}_{12} = k_e \frac{q_1q_2}{r^2}\hat{r}_{12}$
\end{center}

\hspace{.5cm}
Where $F_e$ is the electrostatic force in Newtons, $k_e$ is Coulomb's constant, $q_1$ and $q_2$ are the charges of the objects in Coulombs,
$r$ is the distance between the objects in meters, and $\hat{r}_{12}$ is a unit vector pointing from object 1 to object 2.
Notice the similarities to Newton's law of Gravitation.\\

