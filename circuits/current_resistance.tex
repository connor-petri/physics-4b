\subsection{Current and Resistance}

\hrulefill

\subsubsection*{Current}

\paragraph*{Definition}
Current is defined as the time rate of change of charge through an object. It is measured in amps.

\begin{equation*}
    I=\frac{dQ}{dt} \Rightarrow I = nqv_dA
\end{equation*}

Where $I$ is the current in amps $[A] = [\frac{C}{s}]$, $n$ is the free electron density in electrons/$m^3$, 
$v_d$ is the drift velocity in $m/s$, and $A$ is the cross-sectional area of the conductor in $m^2$.


\paragraph*{Drift Velocity}
Electrons are bouncing around randomly. When an electric field is applied, the bouncing is directed 
in a direction, but it is still chaotic. This bouncing results in heat being generated. Heat is defined 
as the kinetic energy of a particle. The speed of the drift of the electrons is called the drift velocity $(v_d)$.

\begin{equation*}
    v_d = \frac{I_{avg}}{nqA} = \frac{I}{nqA}
\end{equation*}

Where $v_d$ is the drift current, $n$ is the free electron density in $\frac{g}{m^2}$, $q$ is the charge of the current carrier 
(usually an electron) in $C$, and A is the cross-sectional area of the conductor in $m^2$.


\paragraph*{Current Density}
Current density is the current per unit area. It can be calculated using the following formula:

\begin{equation*}
    J = \frac{I}{A} = \sigma A = nqv_d
\end{equation*}

Where $J$ is the current density in $Amps/m^2$, $I$ is the current in amps, $A$ is the cross-sectional 
area of the conductor in $m^2$, $\sigma$ is the conductivity of the material, and $n$ is the free electron 
density in electrons/$m^3$.

\pagebreak

Voltage can be calculated as a function of current density and conductivity as follows:

\begin{equation*}
    \Delta V = E\ell = \frac{\ell J}{\sigma}
\end{equation*}

Where $\Delta V$ is the voltage in volts, $E$ is the electric field in volts per meter, $\ell$ is the length of the conductor in meters,


\hrulefill


\subsubsection*{Resistance}

\paragraph*{Definition}
Resistance is defined as the ratio of voltage to current, also known as Ohm's law. It is measured in ohms.

\begin{equation*}
    R = \frac{\Delta V}{I}
\end{equation*}

Where $R$ is the resistance in $\Omega$, $\Delta V$ is the voltage in volts, and $I$ is the current in $A$.


\paragraph*{Resistivity}
Resistivity is the fundimental property of a material that determines how much it resists the flow of current. 
It is measured in ohm-meters $[\Omega m]$.

\begin{equation*}
    \rho = \frac{1}{\sigma} = \frac{RA}{\ell} \Longrightarrow R = \rho \frac{\ell}{A}
\end{equation*}

Where $\rho$ is resistivity in $\Omega m$, $\sigma$ is conductivity in $S/m$, $R$ is resistance in $\Omega$, $A$ is the cross-sectional area of the 
conductor in $m^2$, and $\ell$ is the length of the conductor in meters.\\

\paragraph*{Conductivity}
Conductivity is the inverse of resistivity. It is measured in Siemens per meter $[\frac{S}{m}]$. $[S] = [\frac{1}{\Omega}]$

\begin{equation*}
    \sigma = \frac{1}{\rho} = \frac{\ell}{RA} \Longrightarrow R = \frac{\ell}{\sigma A}
\end{equation*}

Where $\rho$ is resistivity, $\sigma$ is conductivity, R is resistance in $\Omega$, and $\ell$ is the length of the conductor in meters.

\paragraph*{Ohmic vs. Non-Ohmic devices}
Ohmic devices are devices that have a Voltage vs Current slope of $\frac{1}{R}$. Non-ohmic devices have a slope that changes with voltage or current.

\paragraph*{Resistivity and Temperature}
The resistivity of a material changes with the temperature of the material. The equation for resistivity as a function of temperature is given by:

\begin{align*}
    \rho_t = \rho_0[1+\alpha\Delta T]\\
    \alpha = \frac{\Delta \rho/\rho_0}{\Delta T}
\end{align*}

Where $\rho_t$ is the resistivity of the material at a given temperature in $\Omega m$, $\rho_0$ is the resistivity of the material at a reference temperature in $\Omega m$, 
$\alpha$ is the temperature coefficient in $\degree C^{-1}$ (a material constant, much like $\kappa$), $\Delta \rho$ is the change in resistivity in $\Omega m$, 
and $\Delta T$ is the change in temperature in $\degree C$.

\hrulefill

\subsubsection*{Electrical Power}


\paragraph*{Definition}
Power is defined as the rate at which energy is transferred or converted. It is measured in watts.

\begin{align*}
    P = \frac{dU_e}{dt} = I^2R = \frac{(\Delta V)^2}{R} = IV
\end{align*}

Where $P$ is power in watts, $U_e$ is electric potential energy in Joules, $\Delta V$ is voltage in volts, and $I$ is current in amps.

