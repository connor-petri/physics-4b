\subsection{Direct Current Circuits}
\hrulefill

\subsubsection*{Electromotive Force}

\begin{figure*}[h]
    \centering
    \begin{circuitikz}
        \draw (0,0) node[circ]{} node[label=above:$-$]{} -- (1,0)
            (2,0) to[battery, l=$\varepsilon$] (1,0) 
            (2,0) to[short, i>=$I$] (3,0) to[R, l=$r$] (4,0) -- (5,0) node[circ]{} node[label=above:$+$]{}
            ;
    \end{circuitikz}
    \caption*{Circuit Diagram of a Simple Battery}
\end{figure*}


\paragraph*{Definition}
Electromotive force (emf) is the voltage produced by a battery or other voltage source. It is measured in volts. 
A simple battery consists of a source of emf and a resistor to represent the internal resistance of the battery.
Unless the battery is ideal, $\Delta V_{terminal} \neq \varepsilon$. In this case, we can find the voltage at the terminals of 
the battery using the following equation.

\begin{align*}
    \Delta V_{terminal} &= \varepsilon - Ir\\
    \varepsilon &= \Delta V_{terminal} + Ir
\end{align*}

Where $\Delta V_{terminal}$ is the voltage at the terminals of the battery, $\varepsilon$ is the emf of the battery,
I is the current through, and $r$ is the internal resistance of the battery.\\

When a non-ideal battery is in use, we need to represent the current in terms of the emf and the resistors in the circuit.
The current and emf in the circuit can be found using the following equations.

\begin{align*}
    I &= \frac{\varepsilon}{R+r}\\
    \varepsilon &= IR + Ir
\end{align*}

Where $\varepsilon$ is the emf in volts, $I$ is the current in amps, $r$ is the internal resistance of the battery in ohms,
and $R$ is the resistance of the resistor in ohms.

\begin{figure}[h]
    \centering
    \begin{circuitikz}
        \draw (0,0) node[circ]{} node[label=above:$-$]{} -- (1,0)
            (2,0) to[battery, l=$\varepsilon$] (1,0) 
            (2,0) -- (3,0) to[R, l=$r$] (4,0) -- (5,0) node[circ]{} node[label=above:$+$]{}
            to[short, i>=$I$] (5, -2) -- (3, -2) to[R, l=$R$] (2, -2) -- (0, -2) to[short, i>=$I$] (0, 0)
            ;
    \end{circuitikz}
    \caption*{A Simple Battery wired to a resistor}
\end{figure}

\paragraph*{Power}
The internal resistance also affects power delivery. The power delevered to each of the resistors in the above circuit is given 
by the following equations.

\begin{align*}
    P &= P_R + P_r\\
    P_R &= I^2R\\
    P_r &= I^2r
\end{align*}

Note that the specific current value used will depend on how the resistors in the circuit are connected.

\pagebreak

\subsubsection*{Equivilant Resistance}
\paragraph*{Definition} 
Much like capacitors, multiple resistors can be represented by a single resistor with an equivilant resistance. 
How the resistors are connected will determine how the equivilant resistance is calculated and how the current
and voltage change across them.


\begin{figure}[h]
    \centering
    \begin{subfigure}{0.45\textwidth}
        \centering
        \begin{circuitikz}
            \draw
            (0,0) node[circ]{} node[label=above:$a$]{} -- (1,0) to[R, l=$R_1$] (2,0) 
            -- (3,0) to[R, l=$R_2$] (4,0) -- (5,0) node[label=above:$b$]{} node[circ]{}
            ;
        \end{circuitikz}
        \caption*{Two resistors in series}
    \end{subfigure}%
    \begin{subfigure}{0.45\textwidth}
        \centering
        \begin{circuitikz}
            \draw
            (0,0) node[circ]{} node[label=above:$a$]{} -- (0.5,0)
            (0.5,0) -- (0.5, 1) to[R, l=$R_1$] (2.5, 1) -- (2.5, 0) -- (3,0) node[label=above:$b$]{} node[circ]{}
            (0.5,0) -- (0.5, -1) to[R, l=$R_2$] (2.5, -1) -- (2.5, 0)
            ;
        \end{circuitikz}
        \caption*{Two resistors in parallel}
    \end{subfigure}
\end{figure}


\paragraph*{In Series}
Resistors in series can be combined into a single resistor with a resistance equal to the sum of the individual resistances.
The current through each resistor in series is the same, but the voltage across each resistor is different.

\begin{align*}
    R_{eq(s)} &= \sum_{i=1}^{n} R_i\\
    I_{eq(s)} &= I_1 = I_2 = \cdots = I_n\\
    \Delta V_{eq(s)} &= \sum_{i=1}^{n} \Delta V_i
\end{align*}

\pagebreak

\paragraph*{In Parallel}
Resistors in parallel can be combined into a single resistor with a resistance equal to the reciprocal of the sum of the reciprocals 
of the individual resistances. The voltage across each resistor in parallel is the same, but the current through each resistor is different.


\begin{align*}
    \frac{1}{R_{eq(\parallel)}} &= \sum_{i=1}^{n} \frac{1}{R_i}\\
    I_{eq(\parallel)} &= \sum_{i=1}^{n} I_i\\
    \Delta V_{eq(\parallel)} &= \Delta V_1 = \Delta V_2 = \cdots = \Delta V_n
\end{align*}

\hrulefill


\subsubsection*{Kirchhoff's Laws}

Kirchhoff's Laws are fundamental tools used to analyze electrical circuits. They consist of two rules: the Junction Rule and the Loop Rule. These laws help us determine the current and voltage at different points in a circuit.

\paragraph*{Junction Rule}
The Junction Rule states that at any junction (or node) in a circuit, the sum of the currents entering the junction must equal the sum of the currents leaving the junction. This is a consequence of the conservation of charge.

\begin{equation*}
    \sum_{junction} I = 0 \quad \Longrightarrow \quad \sum_{junction} I_{in} = \sum_{junction} I_{out}
\end{equation*}\\

\paragraph*{To apply the Junction Rule:}
\begin{enumerate}
    \item Identify all junctions in the circuit.
    \item Write an equation for each junction, setting the sum of currents entering the junction equal to the sum of currents leaving the junction.
    \item Solve the system of equations to find the unknown currents.
\end{enumerate}

\paragraph*{Loop Rule}
The Loop Rule states that the sum of the voltages around any closed loop in a circuit must equal zero. This is a consequence of the conservation of energy.

\begin{equation*}
    \sum_{closed loop} \Delta V = 0
\end{equation*}\\

\paragraph*{To apply the Loop Rule:}
\begin{enumerate}
    \item Identify all independent loops in the circuit.
    \item Choose a direction to traverse each loop (clockwise or counterclockwise).
    \item Write an equation for each loop, summing the voltage drops (negative) and voltage rises (positive) around the loop and setting the sum equal to zero.
    \item Solve the system of equations to find the unknown voltages and currents.
\end{enumerate}

\ \ \ \ These rules can be used together to solve for the currents and voltages at different points in a circuit.

\hrulefill

\paragraph*{Steps to Solve a Circuit with Kirchhoff's Laws:}
\begin{enumerate}
    \item Label all currents and voltages: Assign a variable to each unknown current and voltage in the circuit.
    \item Apply the Junction Rule: Write equations for each junction in the circuit.
    \item Apply the Loop Rule: Write equations for each independent loop in the circuit.
    \item Solve the system of equations: Use algebraic methods to solve the system of equations obtained from the Junction and Loop Rules.
\end{enumerate}

\paragraph*{Sign Conventions}
When applying Kirchhoff's Laws, it is important to follow these sign conventions:
\begin{itemize}
    \item If the current enters the positive terminal of a resistor, the voltage drop across the resistor is positive.
    \item If the current enters the negative terminal of a resistor, the voltage drop across the resistor is negative.
    \item If the current enters the positive terminal of a battery, the voltage rise across the battery is negative.
    \item If the current enters the negative terminal of a battery, the voltage rise across the battery is positive.
\end{itemize}

\hrulefill

\subsubsection*{RC Circuits}

\paragraph*{Definition}
An RC circuit is a circuit that contains a resistor and a capacitor. The capacitor stores energy in the form of an electric field, 
while the resistor dissipates energy in the form of heat. The time constant $\tau$ of an RC circuit is given by the product of the 
total resistance and capacitance in the circuit.

\begin{equation*}
    \tau = RC
\end{equation*}

Where $\tau$ is the time constant in seconds, $R$ is the resistance in ohms, and $C$ is the capacitance in farads. 

\paragraph*{Charging a Capacitor}

\begin{align*}
    Q_{max} &= C\varepsilon\\
    I &= \frac{dq}{dt} = -\frac{q-C\varepsilon}{RC}\\
    q(t) &= C\varepsilon(1-e^{-\frac{t}{RC}}) = Q_{max}(1-e^{-\frac{t}{RC}})\\
    i(t) &= \frac{dq}{dt} = \frac{\varepsilon}{R}e^{-\frac{t}{RC}}\\
\end{align*}

\paragraph*{Discharging a Capacitor}
\begin{align*}
    q(t) &= Q_ie^{-\frac{t}{RC}}\\
    i(t) &= -\frac{Q_i}{RC}e^{-\frac{t}{RC}}\\
\end{align*}

