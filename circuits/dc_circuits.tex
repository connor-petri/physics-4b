\subsection*{Direct Current Circuits}
\hrulefill

\subsubsection*{Electromotive Force}
\paragraph*{Definition}
Electromotive force (emf) is the voltage produced by a battery or generator. It is measured in volts. 
The emf of a circuit is the total voltage supplied by the battery or generator.

\begin{align*}
    \Delta V &= \varepsilon - Ir\\
    \varepsilon &= IR + Ir\\
    I &= \frac{\varepsilon}{R+r}
\end{align*}

Where $\Delta V$ is the voltage in volts, $\varepsilon$ is the emf in volts, $I$ is the current in amps, 
and $R$ and $r$ are resistors in ohms.

\begin{figure}[h]
\centering
\begin{circuitikz}
    \draw (1,0) to[battery, l=$\varepsilon$] (0,0)
    (1,0) -- (2,0) to[R, l=$r$] (3,0) -- (4,0)
    -- (4,-2) -- (3,-2) to[R, l=$R$] (1,-2) -- (0,-2) -- (0,0)
    ;
\end{circuitikz}
\end{figure}

\paragraph*{Power}
:

\begin{align*}
    P_R &= I^2R\\
    P_r &= I^2r
\end{align*}

\subsubsection*{Equivilant Resistance}

\ \ \ \ Much like capacitors, multiple resistors can be represented by a single resistor with an equivilant resistance. 

\paragraph*{In Series}

\begin{align*}
    R_{eq(s)} &= \sum_{i=1}^{n} R_i\\
    I_{eq(s)} &= I_1 = I_2 = \cdots = I_n\\
    \Delta V_{eq(s)} &= \sum_{i=1}^{n} \Delta V_i
\end{align*}

\paragraph*{In Parallel}
\begin{align*}
    \frac{1}{R_{eq(\parallel)}} &= \sum_{i=1}^{n} \frac{1}{R_i}\\
    I_{eq(\parallel)} &= \sum_{i=1}^{n} I_i\\
    \Delta V_{eq(\parallel)} &= \Delta V_1 = \Delta V_2 = \cdots = \Delta V_n
\end{align*}


\subsubsection*{Kirchhoff's Laws}

\paragraph*{Junction Rule}
At any junction in a circuit, the sum of the currents must equal zero.

\begin{equation*}
    \sum_{junction} I = 0 \quad \Longrightarrow \quad \sum_{junction} I_{in} = \sum_{junction} I_{out}
\end{equation*}

\paragraph*{Loop Rule}
The sum of the voltages around any closed loop in a circuit must equal zero.

\begin{equation*}
    \sum_{closed loop} \Delta V = 0
\end{equation*}

\paragraph*{Sign Conventions}
\begin{itemize}
    \item If the current enters the positive terminal of a resistor, it is positive.
    \item If the current enters the negative terminal of a resistor, it is negative.
    \item If the current enters the positive terminal of a battery, it is negative.
    \item If the current enters the negative terminal of a battery, it is positive.
\end{itemize}

\hrulefill

\paragraph*{Charging a Capacitor}

\begin{align*}
    Q_{max} &= C\varepsilon\\
    I &= \frac{dq}{dt} = -\frac{q-C\varepsilon}{RC}\\
    q(t) &= C\varepsilon(1-e^{-\frac{t}{RC}}) = Q_{max}(1-e^{-\frac{t}{RC}})\\
    i(t) &= \frac{dq}{dt} = \frac{\varepsilon}{R}e^{-\frac{t}{RC}}\\
\end{align*}


\paragraph*{Discharging a Capacitor}
\begin{align*}
    q(t) &= Q_ie^{-\frac{t}{RC}}\\
    i(t) &= -\frac{Q_i}{RC}e^{-\frac{t}{RC}}\\
\end{align*}

