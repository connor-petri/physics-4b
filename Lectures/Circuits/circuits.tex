\documentclass[12pt]{article}
\usepackage{graphicx} % Required for inserting images
\usepackage{amsmath, amssymb, amsthm}

\title{
    PHYS4B Electromagnitism for Scientists and Engineers:\\
    \vspace{12pt}
    Circuits
}
\author{Connor Petri}
\date{Winter 2025}

\begin{document}

\maketitle
\tableofcontents
\pagebreak



\begin{center}
\section{Current and Resistance}
\hrulefill
\subsection*{Current}
\end{center}

\subsubsection*{Definition}
\hspace{.5cm} Current is defined as the time rate of change of charge through an object. It is measured in amps.

\vbox{
    \large\center
    $I=\frac{dQ}{dt} \Rightarrow I = nqv_dA$
}\vspace{12pt}

Where $I$ is the current in amps, $n$ is the free electron density in electrons/$m^3$, 
$v_d$ is the drift velocity in $m/s$, and $A$ is the cross-sectional area of the conductor in $m^2$.

\vbox {
    \large\center
    $[Amps] = [\frac{Coulombs}{second}]$
}


\subsubsection*{Drift Velocity}
Electrons are bouncing around randomly. When an electric field is applied, the bouncing is directed 
in a direction, but it is still chaotic. This bouncing results in heat being generated. Heat is defined 
as the kinetic energy of a particle. The speed of the drift of the electrons is called the drift velocity $(v_d)$.

\vbox{
    \large\center
    $v_d = \frac{I_{avg}}{nqA} = \frac{I}{nqA}$\\
}\vspace{12pt}

Where $v_d$ is the drift current, n is the free electron density, q is the charge of the current carrier 
(usually an electron) and A is the cross-sectional area of the conductor.

\pagebreak



\subsubsection*{Current Density}
\hspace{.5cm} Current density is the current per unit area. It is defined as:\\
\vbox{
    \large\center
    $J = \frac{I}{A} = \sigma A = nqv_d$
}\vspace{12pt}

Where $J$ is the current density in $Amps/m^2$, $I$ is the current in amps, $A$ is the cross-sectional 
area of the conductor in $m^2$, $\sigma$ is the conductivity of the material, and $n$ is the free electron 
density in electrons/$m^3$.\\

Voltage can be calculated as a function of current density and conductivity as follows:\\
\vbox{
    \large\center
    $\Delta V = E\ell = \frac{\ell J}{\sigma}$
}
\vspace{12pt}
\hrulefill

\begin{center}
\subsection*{Resistance}
\end{center}

\subsubsection*{Definition}
Resistance is defined as the ratio of voltage to current, also known as Ohm's law. It is measured in ohms.\\

\vbox{
    \large\center
    $R = \frac{\Delta V}{I}$
}\vspace{12pt}

Where $R$ is the resistance in ohms, $\Delta V$ is the voltage in volts, and $I$ is the current in amps.\\


\subsubsection*{Resistivity}

\hspace{.5cm} Resistivity is the fundimental property of a material that determines how much it resists the flow of current. 
It is measured in ohm-meters.

\vbox {
    \large\center
    $\rho = \frac{1}{\sigma} \Rightarrow R = \rho \frac{\ell}{A}$
}\vspace{12pt}

Where $\rho$ is resistivity, $\sigma$ is conductivity, R is resistance in $\Omega$, $I$ is current in amps, and $\Delta V$ is voltage in volts.

\vbox {
    \large\center
    $[\Omega] = [\frac{V}{A}]$
}\vspace{12pt}

\subsubsection*{Ohmic vs. Non-Ohmic devices}
\hspace{.5cm} Ohmic devices are devices that have a Voltage vs Current slope of $\frac{1}{R}$. Non-ohmic devices have a slope that changes with voltage or current.\\

\subsubsection*{Resistance and Temperature}
\vbox{
    \large\center
    $\rho = \rho_0[1+\alpha (T-T_0)]$\\
    \vspace{12pt}
    $\alpha = \frac{\Delta \rho/\rho}{\Delta T}$
}
\vspace{12pt}

Where $\rho$ is resistivity and $T$ is temperature.

\vspace{12pt}
\hrulefill

\begin{center}
\subsection*{Electrical Power}
\end{center}

\subsubsection*{Definition}
\hspace{.5cm} Power is defined as the rate at which energy is transferred or converted. It is measured in watts.\\
\vbox{
    \large\center
    $P = \frac{dU_e}{dt} = \frac{d}{dt} (Q\Delta V) = \frac{dQ}{dt}\Delta V = I\Delta V$
}

\vbox{
    \large\center
    $\Longrightarrow P = I^2R = \frac{(\Delta V)^2}{R}$
}
\vspace{12pt}

Where $P$ is power in watts, $Q$ is total charge in Coulombs, $\Delta V$ is voltage in volts, and $I$ is current in amps.


\end{document}
