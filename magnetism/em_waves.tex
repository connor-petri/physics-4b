\subsection{Electromagnetic Waves}
\hrulefill

\paragraph*{Definition}
Electromagnetic waves are waves of electric and magnetic fields that propagate through space. They are generated by accelerating charges 
and are characterized by their frequency and wavelength. Electromagnetic waves are governed by Maxwell's equations, which describe how electric
and magnetic fields interact with each other.

\paragraph*{Maxwell's Equations}

\begin{align*}
    \oint \vec{E} \cdot d\vec{A} &= \frac{Q_{enc}}{\varepsilon_0} &\text{Gauss's Law}\\
    \oint \vec{B} \cdot d\vec{A} &= 0 &\text{Gauss's Law in Magnetism}\\
    \oint \vec{E} \cdot d\vec{\ell} &= -\frac{d\Phi_B}{dt} &\text{Faraday's Law}\\
    \oint \vec{B} \cdot d\vec{\ell} &= \mu_0 I + \varepsilon_0\mu_0\frac{d\Phi_E}{dt} &\text{Ampere-Maxwell Law}
\end{align*}

Where $\vec{E}$ is the electric field in volts per meter, $\vec{B}$ is the magnetic field in teslas, $d\vec{A}$ is an infinitesimal 
segment of the circuit in meters, $Q_{enc}$ is the charge enclosed by the loop in coulombs, $\varepsilon_0$ is the permittivity of free 
space in farads per meter, $d\vec{\ell}$ represents an infinitesimal segment of the wire's length, $\Phi_B$ is the magnetic flux in webers, 
$\mu_0$ is the permeability of free space in henries per meter, $I$ is the current in amperes, and $\Phi_E$ is the electric flux in webers.\\

Note that if there is no current, $\mu_0 I = 0$, and the Ampere-Maxwell Law simplifies to:
\begin{align*}
    \oint \vec{B} \cdot d\vec{\ell} &= \varepsilon_0\mu_0\frac{d\Phi_E}{dt}
\end{align*}

\pagebreak

\subsubsection*{Electromagnetic Wave Equation}
The electromagnetic wave equation is a second-order partial differential equation that describes how electromagnetic waves propagate through space.
Assuming that the wave is propagating in the $x$ direction, the wave equation is given by the following system of equations.

\begin{align*}
    \large
\begin{cases}
    \frac{\partial^2\vec{E}}{\partial x^2} &= \mu_0\varepsilon_0\frac{\partial^2\vec{E}}{\partial t^2}\\
    \frac{\partial^2\vec{B}}{\partial x^2} &= \mu_0\varepsilon_0\frac{\partial^2\vec{B}}{\partial t^2}\\
    \frac{\partial^2 \vec{y}}{\partial x^2} &= \frac{1}{v^2}(\frac{\partial^2 \vec{y}}{\partial t^2})
\end{cases}
\end{align*}
    
Where $\vec{E}$ is the electric field in volts per meter, $\vec{B}$ is the magnetic field in teslas, $x$ is the position in meters, $t$ is the time 
in seconds, $\mu_0$ is the permeability of free space in henries per meter, $y$ is the wave function, $x$ is the position in meters and $\varepsilon_0$ 
is the permittivity of free space in farads per meter.\\

\paragraph{The Speed of Light}
The speed of light is the speed at which electromagnetic waves propagate through space. It is given by the following equation when the wave is propogating
through a vaccum.

\begin{align*}
    c_{vac} &= \frac{1}{\sqrt{\mu_0\varepsilon_0}} = 3.00 \times 10^8 \text{ m/s}
\end{align*}

\paragraph*{Solution to the Wave Equation}

\begin{align*}
    \frac{E_{max}}{B_{max}} &= c
\end{align*}

\paragraph{Poynting Vector}
The Poynting vector is a vector that describes the direction and magnitude of the energy flow in an electromagnetic wave. It is given by the following equation.

\begin{align*}
    \vec{S} &= \frac{1}{\mu_0}\vec{E} \times \vec{B}
\end{align*}

Where $\vec{S}$ is the Poynting vector in watts per square meter, $\vec{E}$ is the electric field in volts per meter, $\vec{B}$ is the magnetic field in teslas,
and $\mu_0$ is the permeability of free space in henries per meter.


\paragraph{Energy Carried by Electromagnetic Waves}
The energy carried by an electromagnetic wave is given by the following equation.

\begin{align*}
    U_E &= \frac{1}{2}\varepsilon_0E_{max}^2\\
    U_B &= \frac{1}{2\mu_0}B_{max}^2\\
    I &= S_{avg} = \frac{cB_{max}^2}{2\mu_0} = \frac{c}{U_{avg}}
\end{align*}

\paragraph{Momentum and Radiation Pressure}
Electromagnetic waves carry momentum and exert a pressure on objects they interact with. The momentum carried by an electromagnetic wave is given by the following 
equations.

\begin{align*}
    P &= \frac{S}{c} &\text{(Complete Absorption)}\\
    P &= \frac{2S}{c} &\text{(Complete Reflection)}\\
    P &= (1 + f) \frac{S}{c} &\text{(Partial Absorption)}
\end{align*}

Where $P$ is the momentum in newton-seconds, $S$ is the Poynting vector in watts per square meter, $c$ is the speed of light in meters per second, 
and $f$ is the fraction of the wave that is absorbed (check this one for accuracy).

