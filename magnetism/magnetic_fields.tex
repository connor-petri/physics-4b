\subsection{Magnetic Fields}
\hrulefill

\paragraph*{Definition}
    A magnetic field is a vector field that describes the magnetic influence on moving electric charges, 
electric currents, and magnetic materials. They are generated by moving charges, or current. 
It is represented by the symbol $\vec{B}$ and is measured in teslas (T).\\

    Unlike electric fields, which can be generated by a single source charge, magnetic poles always come in pairs. 
There are no magnetic monopoles. The two types of magnetic poles are called north and south.

\begin{equation*}
    \vec{F}_B = q\vec{v} \times \vec{B}
\end{equation*}

    Where $\vec{F}_B$ is the magnetic force in newtons, $q$ is the charge in coulombs, $\vec{v}$ is the velocity
of the charged in meters per second, and $\vec{B}$ is the magnetic field in teslas.\\

    Note that the magnetic force Insert perpendicular to both the velocity of the charge and the magnetic field. 
This is different from the electric force, which is parallel to the electric field. The direction of the magnetic 
force can be determined using the right hand rule.\\

[Insert right hand rule stuff]

$\odot$ is into to page, $\otimes$ is out of page


\paragraph{Charged Particle in a Uniform Magnetic Field}
When a charged particle moves in a uniform magnetic field, it experiences a constant magnetic force that is directed
radially inward. This causes the particle to move in uniform circular motion. We can revise the UCM equations in terms
of the magnetic field and charge of the particle in UCM.

\begin{align*}
    \vec{F_B} &= qvB = \frac{mv^2}{r}\\
    r &= \frac{mv}{qvB}\\
    \omega &= \frac{v}{r} = \frac{qv}{m}\\
    T &= \frac{2\pi r}{v} = \frac{2\pi}{\omega} = \frac{2\pi m}{qB}\\
\end{align*}


\paragraph*{Magnetic Force Acting on a Current-Carrying Wire}

\begin{align*}
    \vec{F}_B &= I\vec{L} \times \vec{B}\\
\end{align*}

\paragraph*{Torque on a Current Loop in a Magnetic Field}

\begin{align*}
    \vec{\tau}_{mag,dipole} &= \vec{\mu} \times \vec{B}\\
    \vec{\mu} &= I\vec{A}\\
    U_B &= -\vec{\mu} \cdot \vec{B}
\end{align*}


\paragraph*{Hall Effect}
The Hall effect is the production of a voltage difference (the Hall voltage) across an electrical conductor,
when a magnetic field is applied perpendicular to the current. The Hall voltage is given by the following equation.

\begin{align*}
    E_H &= v_dB = \frac{IB}{nq}\\
    \Delta V_H &= E_H d = \frac{IBd}{nqA} = \frac{IB}{nqt}
\end{align*}