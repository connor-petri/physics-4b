\subsection{Sources of Magnetic Fields}
\hrulefill

\paragraph*{Biot-Savart Law}
The Biot-Savart law describes the magnetic field produced by a current-carrying wire. It is given by the following equation.

\begin{align*}
    d\vec{B} &= \frac{\mu_0}{4\pi}\frac{Id\vec{\ell} \times \hat{r}}{r^2}\\
    \vec{B} &= \frac{\mu_0 I}{4\pi}\int \frac{d\vec{\ell} \times \hat{r}}{r^2}\\
\end{align*}

Where $d\vec{B}$ is the magnetic field in teslas, $\mu_0$ is the permeability of free space in henries per meter (a constant), 
$I$ is the current in amperes, $d\vec{\ell}$ represents an infinitesimal segment of the wire's length, $\hat{r}$ is the unit vector 
pointing from the wire to the point where the magnetic field is being measured, and $r$ is the distance from the wire to the point where 
the magnetic field is being measured in meters.\\

Much like the electric field, the magnetic field generated by an object varies with it's geometry. The two most common geometries are a straight 
wire and a loop of wire.

\begin{align*}
    \vec{B}_{straight} &= \frac{\mu_0 I}{2\pi r}\\
    \vec{B}_{loop} &= \frac{\mu_0 I}{2R}\\
    \frac{\vec{F}_{B\parallel}}{\ell} &= \frac{\mu_0 I_1 I_2}{2\pi d}
\end{align*}

Where $\vec{B}_{straight}$ is the magnetic field of a straight wire in teslas, $\vec{B}_{loop}$ is the magnetic field of a loop of wire in teslas,
$\mu_0$ is the permeability of free space in henries per meter, $I$ is the current in amperes, $r$ is the distance from the wire to the point where
the magnetic field is being measured in meters, and $R$ is the radius of the loop in meters.\\


\paragraph*{Ampere's Law}
Ampere's Law is similar to Gauss's Law, but for magnetic fields. Unlike Gauss's Law, Ampere's Law is a line integral rather than an area integral. 
It is given by the following equation.

\begin{align*}
    \oint \vec{B} \cdot d\vec{\ell} = \mu_0 I_{enc}
\end{align*}

Where $\vec{B}$ is the magnetic field in Teslas, $d\vec{\ell}$ represents an infinitesimal segment of the wire's length, $\mu_0$ is the 
permeability of free space in henries per meter, and $I_{enc}$ is the current enclosed by the loop in amperes.\\

\subsubsection*{Toroids and Solenoids}
\paragraph*{Definitions}
A solenoid is a coil of wire that generates a magnetic field when a current is passed through it. A toroid is a solenoid bent into a circle. 

\paragraph*{Magnetic Field of a Solenoid}
The magnetic field generated by a solenoid is given by the following equation.

\begin{align*}
    \vec{B} &= \mu_0 nI\\
    n &= \frac{N}{L}
\end{align*}

Where $\vec{B}$ is the magnetic field in teslas, $\mu_0$ is the permeability of free space in henries per meter, $n$ is the number of turns per
unit length in turns per meter, $I$ is the current in amperes, $N$ is the number of turns in the solenoid, and $L$ is the length of the solenoid in meters.\\

\paragraph*{Magnetic Field of a Toroid}
The magnetic field generated by a toroid is given by the following equation.

\begin{align*}
    \vec{B} &= \frac{\mu_0 NI}{2\pi r}
\end{align*}

Where $\vec{B}$ is the magnetic field in teslas, $\mu_0$ is the permeability of free space in henries per meter, $N$ is the number of turns 
in the toroid,


\subsubsection*{Magnetic Flux}
\paragraph*{Definition}
Magnetic flux is a measure of the number of magnetic field lines that pass through a given area. It is given by the following equation.

\begin{align*}
    \Phi_B &= \int \vec{B} \cdot d\vec{A} = BA\cos(\theta)
\end{align*}

Where $\Phi_B$ is the magnetic flux in webers, $\vec{B}$ is the magnetic field in teslas, and $\vec{A}$ is the area in square meters.\\

Notice the similarity of this equation to the definition of electric flux, just with the magnetic field in place of the electric field.\\

\paragraph*{Gauss's Law for Magnetism}
Gauss's Law for Magnetism is similar to Gauss's Law for Electricity, but for magnetic fields. It states that the magnetic flux through any
surface surrounding a magnetic dipole is zero. This is given by the following equation.

\begin{align*}
    \Phi_B &= \oint \vec{B} \cdot d\vec{A} = 0
\end{align*}

Where $\Phi_B$ is the magnetic flux in webers, $\vec{B}$ is the magnetic field in teslas, and $\vec{A}$ is the area in square meters.\\

\begin{center}
    \includegraphics*[scale=0.5]{bflux.png}
\end{center}
