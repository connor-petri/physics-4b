\subsection{Inductance}
\hrulefill

\paragraph*{Self-Inductance}
Self-inductance is the property of a circuit that causes it to generate an electromotive force in response to a change in current. It is measured in henries $[H] = [V\cdot \frac{s}{A}]$. The electromotive force generated by self-inductance is given by the following equation.

\begin{align*}
    \varepsilon_L &= -L\frac{dI}{dt}\\
    L &= \frac{\varepsilon_L}{dI/dt} = \frac{N\Phi_B}{I}
\end{align*}

Where $\varepsilon_L$ is the electromotive force generated by self inductance in volts, $L$ is the self-inductance in henries, $I$ is the current in amperes, $N$ is the number of turns in the coil, and $\Phi_B$ is the magnetic flux in webers.\\

\subsubsection*{RL Circuits}
\paragraph*{Definition}
An RL circuit is a circuit that contains a resistor and an inductor. When a current is passed through the circuit, the inductor generates
an electromotive force that opposes the change in current. This causes the current to increase more slowly than it would in a circuit with
only a resistor. The current in an RL circuit is given by the following equation.

\begin{align*}
    I(t) &= \frac{\varepsilon}{R}(1 - e^{-t/\tau})\\
    \tau &= \frac{L}{R}\\
    I_{max} &= \frac{\varepsilon}{R} 
\end{align*}

Where $I(t)$ is the current in amperes, $\varepsilon$ is the electromotive force in volts, $R$ is the resistance in ohms, $t$ is the time in seconds,
$\tau$ is the time constant in seconds, $L$ is the inductance in henries, and $I_{max}$ is the maximum current in amperes.\\


\paragraph*{Energy in a Magnetic Field}
The energy stored in a magnetic field is given by the following equation.

\begin{align*}
    U_B &= \frac{1}{2}LI^2
\end{align*}

Where $U_B$ is the energy stored in the magnetic field in joules, $L$ is the inductance in henries, and $I$ is the current in amperes.\\

\paragraph*{Energy in a Solinoid in a Magnetic Field}
\begin{align*}
    u_B = \frac{U_B}{V} = \frac{B^2V}{2\mu_0}
\end{align*}

Where $u_B$ is the energy density in the magnetic field in joules per cubic meter, $U_B$ is the energy stored in the magnetic field in joules, 
$V$ is the volume of the solenoid in cubic meters, $B$ is the magnetic field in teslas, and $\mu_0$ is the permeability of free space in henries per meter.\\


\paragraph*{Mutual Inductance}
Mutual inductance is the property of a circuit that causes it to generate an electromotive force in response to a change in current in a
neighboring circuit. It is measured in henries. 


\begin{align*}
    M_{12} &= \frac{N_2\Phi_12}{I_1} &M_{21} = \frac{N_1\Phi_21}{I_2}\\
    \varepsilon_1 &= -M_{21}\frac{dI_2}{dt} &\varepsilon_2 = -M_{12} \frac{dI_1}{dt}
\end{align*}

Where $M_{12}$ is the mutual inductance from circuit 1 to circuit 2 in henries, $N_2$ is the number of turns in circuit 2, $\Phi_{12}$ is the magnetic flux
from circuit 1 to circuit 2 in webers, $I_1$ is the current in circuit 1 in amperes, $M_{21}$ is the mutual inductance from circuit 2 to circuit 1 in henries
$N_1$ is the number of turns in circuit 1, $\Phi_{21}$ is the magnetic flux from circuit 2 to circuit 1 in webers, $I_2$ is the current in circuit 2 in
amperes, $\varepsilon_1$ is the electromotive force generated in circuit 1 in volts, $\varepsilon_2$ is the electromotive force generated in circuit 2 in
volts, and $M_{12}$ is the mutual inductance from circuit 1 to circuit 2 in henries.\\


\paragraph*{Oscillations in an LC Circuit}
An LC circuit is a circuit that contains a capacitor and an inductor. When a charge is passed through the circuit, the capacitor stores energy in the form of 
an electric field. When the capacitor is fully charged, the inductor generates an electromotive force that causes the charge to flow in the opposite direction.
This causes the capacitor to discharge, and the cycle repeats. Note that this cycle is between electric potential energy and magnetic poteneial
energy. The charge in an LC circuit is given by the following equation.

\begin{align*}
    q(t) &= Q_{max}\cos(\omega t + \phi)\\
    \omega &= \frac{1}{\sqrt{LC}}\\
    I &= \frac{dq}{dt} = -\omega Q_{max}\sin(\omega t + \phi) = -I_{max}\sin(\omega t + \phi)
\end{align*}

Where $q(t)$ is the charge in coulombs, $q_{max}$ is the maximum charge in coulombs, $\omega$ is the angular frequency in radians per second, $L$ is the 
inductance in henries, $C$ is the capacitance in farads, and $t$ is the time in seconds.\\

\subsubsection*{RLC Circuits}
An RLC circuit is a circuit that contains a resistor, an inductor, and a capacitor. When a charge is passed through the circuit, the capacitor stores energy 
in the form of an electric field. When the capacitor is fully charged, the inductor generates an electromotive force that causes the charge to flow in the 
opposite direction. This causes the capacitor to discharge, and the cycle repeats. The energy model for this circuit is given by the following equation.

\begin{align*}
    \Delta U_E + \Delta U_B + \Delta \varepsilon_{int} &= 0\\
    L\frac{d^2q}{dt^2} + R\frac{dq}{dt} + \frac{q}{C} &= 0
\end{align*}

Where $\Delta U_E$ is the change in electric potential energy, $\Delta U_B$ is the change in magnetic potential energy, and $\Delta \varepsilon_{int}$ is the change in internal energy.\\

The charge in an RLC circuit is given by the following equation.
\begin{align*}
    q(t) &= q_{max}e^{-\frac{R}{2L}t}\cos(\omega_d t + \phi)\\
    \omega_d &= \sqrt{\frac{1}{LC} - \frac{R^2}{4L^2}}
\end{align*}

Where $q(t)$ is the charge in coulombs, $q_{max}$ is the maximum charge in coulombs, $R$ is the resistance in ohms, $L$ is the inductance in henries, $C$ is the capacitance in farads, 
$\omega_d$ is the damped angular frequency in radians per second, and $t$ is the time in seconds.\\